\section{Introduction}
With the explosion of the data, the industry has set its sight on distributed computing systems for their data processing problems. However, this field has several technical challenges and problems to solve. For instance, what happens when part of the system that was executing some work crashes? How to distribute workload across the system in a reasonable balanced manner? To tackle this scenario, WantDS BV has the goal of designing and implementing a distributed simulator of a multi-cluster system and study its feasibility. 
%This paragraphs answers: describe the problem
\\\\
The distributed system designed is formed by several grids of independent clusters of nodes that carry out user workloads, consisting on sequential stateless asynchronous jobs of a fixed duration. Each grid is monitored and controlled by one or more grid scheduler (GS).
% this one: the system you are about to implement
\\\\
This document describes the process of designing, developing and evaluating such system. Section 2 details the requirements of the proposed Virtual Grid System (VGS). Section 3 analyzes the considered design approaches and compares their respective merits and deficiencies. In section 4 the most important details of the implementation are briefly explained. Subsequently, section 5 shows the results of the experiments conducted to evaluate and assess the system. Finally, the designed and implemented system and its trade-offs are discussed in Section 6, followed by a conclusion that contains the verdict whether the system is feasible or not.
%the structure of the remainder of the article