\section{System design}
In this chapter, the design of the system is explained.
The focus of section 3.1 will be the overview of the implemented distributed system.
Here, the architecture of the system is discussed, which fault tolerance mechanism is used and how the system replicates the workload across multiple distributed clusters.

\subsection{System overview}


\subsubsection{Topology}
The proposed architecture consists of a logical ring of GSs and evenly divided groups of RMs.
Each group of RMs is connected primarily to one GS and the next GS in the ring as backup. This topology has been chosen because of its 
%its != it's
simplicity because it guarantees that every RM has redundant connections with more than one GS. Also, every GS has to synchronize its job status with another GS (the next in the ring). Thus, in case of GS/RM failure the workload is not lost.

Here we could explain a little bit more about the topology characteriscitcs
 

\subsubsection{Protocol}
I guess here we will have to explain which messages we use and how to process them right? 

\subsubsection{Status}
Every GS sends periodic heartbeats in either direction in the ring so every GS is aware of the status of the rest of GSs. Also, % When offline? 1 missed? 3 missed?
every RM sends periodic heartbeats to both GSs which is connected to.
The heartbeat from RM to GS also carries an indication of that RM's load and its status.

\subsubsection{Load balancing}
When a GS receives a job offloaded from a RM it selects a new RM according to the load of the other GSs. This information about the load is received on the heartbeats sent periodically.

\subsubsection{Job execution}
When a client sends a job to an RM the RM sends the job to its GS for tracking or to offload it. Subsequently, the GS sends the job information to the next GS in the for replication.
This backup GS will keep track of the job until its completion.
Once a job is accepted by an RM it will be scheduled in an available node following a FIFO discipline. On completion, the RM will send the results to the user and notify both GS about it before starting with new jobs.

\subsubsection{Fault recovery}
Failures of RM are discovered by the GSs, either the main one or the backup. Unless the main GS of that RM had also crashed, the failures in RM will be discovered by the main GS. This is due to the fact that the RM sends heartbeats with a higher interval to the backup GS for the sake of overload. When a RM is not sending the periodical heartbeats for a certain amount of time, the main GS will assume that it is down. Therefore, it will reallocate the jobs that were running on that cluster in another one. Also, the backup GS will be notified so it can stop tracking that RM. The GS controlling the cluster selected to run that workload will be notified to keep track of it as on a normal execution.
\\
In the case that the backup GS discovers that the main GS has crashed but the RM is still running it will notify an additional GS as backup till the execution of that job is completed. Also, if both GS or RM have crashed it will relaunch the job in another cluster, selecting a proper backup for it.
\subsubsection{Fault tolerance}
As discussed above, the system design tolerates single failure of both GS and RM. In the case of simultaneous failures of two or more GS/RM the design would not be valid as that workload will be lost. However, we assume that the probability of occurring this is low and that the nodes reboot quickly. 

% Needs to be more complete:
% describe assigning backup separately
% describe promoting backup separately
% Should be in flow diagrams