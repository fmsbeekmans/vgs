\documentclass{article}
\usepackage[utf8]{inputenc}
\usepackage{graphicx}
\usepackage[isolatin]{inputenc}
%\usepackage[spanish]{babel}
\usepackage{amsfonts}
\usepackage{amsmath}
\usepackage{amssymb}
\usepackage{bm}
\usepackage{multirow}
\usepackage{color}

\title{Assignment I: Performance of Polar Codes}
\author{Santiago Conde Camacho}
\date{May 2015}

\begin{document}

\maketitle

\section{Introduction}
Santi will write the Background on application and system design (4, 5a). 
\\\\
In this assignment the behavior of several polar codes will be explored, constructed for one particular value of \textsc{snr} but applied to a finite range of channel \textsc{snr} rather than a single one.
\section{Setup}
Following the professor's advice, the Arikan algorithm \cite{arikan01} will be used to create the different polar codes as it is one of the easiest possible methods. Nonetheless, a slight modification is needed due to the non-universality of polar codes. The original recursive algorithm requires an initial value of 0.5, which corresponds to the worst \textsc{ber}. It will be replaced with exp($R E_b/N_0$) where $R E_b/N_0$ stands for \textit{design} \textsc{snr}. 
\\\\
The channels are \textsc{awgn} so, in order to estimate the \textsc{ber}, it is possible to apply the Gaussian approximation [2]. It basically consists on: 
\begin{enumerate}
\item Calculate the mean of the likelihoods under the successive decoding algorithm.
\item Estimate the noise variance for every channel.
\item Calculate the upper and lower bound of the bit error rate.
\end{enumerate}
\\
The implementation code of this entire process is provided in the attached documents. 
\section{Results}
The code parameters ($N$, $K$ and $R$) are not specified on the assignment proposal, just the \textit{design} \textsc{snr} and the range to plot the results. The examples in the literature [3] show results for $R=0.5$ therefore, that rate will be used on the following simulations. Results for $N=16$ and $K=8$ are displayed in figure 1.

\begin{center}
 \includegraphics[scale=0.35]{16_8.png}
 \centerline{Figure 1: Simulation results for $N=16$, $K=8$}
\end{center}
\vspace{0.25cm}
Despite that all the polar codes for the range \textsc{snr} $=$ $0$, $1$,..., $10$ dB have been plotted, only two graphs are shown on the figure. This is due to the fact that for the parameters used on the simulation, the frozen indices (and consequently the channels used) are the same for every code. Thus, the results obtained are equal for all the codes and it is only possible to distinguish the upper and lower bounds corresponding with the last simulation ($10$ dB code).
\\\\
A simulation for $N=512$ and $K=256$ can be seen in figure 2. It is possible to identify a greater number of graphs. The explanation is that now the frozen indices are not equal for each code. However, we can realize that the code achieving the best results is the one corresponding with \textsc{snr} $= 10$ dB. This behavior is not normal because, the graph should start to go down when it get closer to its \textit{design} \textsc{snr} and not before. On the other hand, it is noticeable that as we increase the length of the codewords, the bit error rate estimation improves. This is because the polarization effect, with a higher value of $N$, more bits tend to polarize and the results are better. In figure 1 the best value was for $E_b/N_0=11$ dB \approx $ 2$\cdot $10^{-9}$, in figure 2 for $E_b/N_0=10$ dB \approx $ 1.25$\cdot $10^{-35}$ (both results for the $10$ dB code).

\begin{center}
 \includegraphics[scale=0.35]{512_256.png}
 \centerline{Figure 2: Simulation results for $N=512$, $K=256$}
\end{center}
\vspace{0.25cm}
Next, in figure 3 we can see the same simulation but for $K=384$ in this case. This shows that, the more channels are used for the same length of the codewords, the worse results we get. The explanation is that when transmitting through a great number of channels it is not possible to choose only those one with highest $I(W_N ^{(j)})$ thus, the results are worse.
\begin{center}
 \includegraphics[scale=0.35]{512_384.png}
 \centerline{Figure 3: Simulation results for $N=512$, $K=384$}
\end{center}  

\vspace{0.25cm}
Finally, another interesting effect will be explained. If we move further on the \textit{design} \textsc{snr} (e.g: $15$, $16$,..., $30$ dB) we realized that the aforementioned strange phenomenon does not occur anymore. In this new scenario, the graph which drops the last (the one achieving worst results) is the one with the highest \textsc{snr} (as expected). Apart from that, it is noticeable that the farther we move on the  $E_b/N_0$ scale, the best results we get for every code. For the same construction parameters than in figure 2 we now reach for $E_b/N_0=12$ dB \approx $ 9.05$\cdot $10^{-75}$ (in this case for the $15$ dB code, the lowest one).

\begin{center}
 \includegraphics[scale=0.35]{512_256_b.png}
 \centerline{Figure 4: Another simulation for $N=512$, $K=256$}
\end{center}  
\\\\
\section{Conclusions}
There is no doubt that polar codes have some features that makes them interesting from an academic point of view. They are capacity achieving codes, that means they reach $R_n \to C$.
However, they lack universality. In this assignment, I have tried to study their estimation bit error rate when applied to a range of channel \textsc{snr} rather than to a single value. The results achieved are good but some strange phenomena are present. When the \textit{design} \textsc{snr} is low (e.g: $0$, $1$,..., $10$ dB) the code achieving the best results is the one with the highest \textit{design} \textsc{snr}. Taking into account the results from the literature, it should be the opposite. Nonetheless, if we create polar codes for a higher \textit{design} \textsc{snr}, they behave as expected. This is, the code with the lowest \textsc{snr} achieves the best results. On the other hand, these results vary strongly with the $E_b/N_0$ range where we plot the results.
\begin{thebibliography}{9}

\bibitem{arikan01}
  E. Arıkan. \textit{"Channel polarization: a method for constructing capacity-achieving codes for symmetric binary-input memoryless channels"}. IEEE Trans. Inf. Theory, 55(7): 3051–3073, 2009.
\bibitem{yuan02}
H. Li, J. Yuan. \textit{"A practical construction method por polar codes in AWGN channels"}. In Proc. TENCON'13, 2013.
\bibitem{other03}
  E. Arıkan. \textit{"Systematic Polar Coding"}. IEEE Communications letters, 15(8): 860-862, 2011.
\end{thebibliography}
\end{document}
