\section{Background on application}
The Virtual Grid Simulator (VGS) is a tool to monitor how a simulated distributed system performs conducting user workloads. There are two types of components: resource managers (RM), which are in control of a cluster of nodes and grid schedulers (GS), which monitor and control the different resource managers. Users send their workload (jobs) to the cluster of their selection. If the selected cluster is free (there are nodes available to perform the work), it will accept the job adding it to the local queue. In case that the queue is full, the RM will notify the GS. The latter will maintain another queue to allocate the jobs as soon as possible, either on the same cluster or a different one. Once the job is finished, the RM will send the result to the requester user and release the resources. Once a RM receives a job from the GS, it has to accept it, even if that means adding it to the local waiting queue. 
%describe the VGS application
\\\\
To enable the study in realistic environments, the system will consist on 5 grid schedulers (GS) and 20 clusters containing 1000 nodes each. On top of each cluster there is a resource manager (RM), thus, there are in total 20 RM. The minimum workload the system supports is 10,000 asynchronous stateless jobs of fixed duration. Each job records the identifier of the cluster when it originally arrives. In the case that job is reallocated to another cluster it will add the identifier of all the cluster where it passes.
%scalability
\\\\
The system is resilient to single failures. That is, the system is able to operate normally even if there is a failure in a grid scheduler node or in a resource manager node. However, to increase the availability of the system and its tolerance to failures, WantDS also covered the failure of two nodes, either two GS or one RM and one GS. The user is not aware of these failures, her workload is carried out transparently. Additionally, to make debugging of the system easier all the events and the other they occur will be stored in two grid schedulers.
%fault tolerance
\\\\
Lastly, the system will handle with load imbalance. The ratio of the jobs arriving at the most and least loaded cluster the system is able to tolerate is 5.
%load balance