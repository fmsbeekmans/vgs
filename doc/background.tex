\section{Virtual Grid System}
The Virtual Grid System (VGS) is a tool to monitor how a simulated distributed multi-grid system performs conducting user workloads. Each cluster contains two types of resources: processors (nodes that carry out jobs) and a resource manager (RM) which controls and monitors the cluster. For grid operation, such a set of clusters needs a grid scheduler (GS), which enables load balancing across the clusters and oversees the correct execution of the jobs. Users send their workload (jobs) to the cluster of their choice. Depending on whether the cluster has resources available to attend the request or not, the job will be added to the cluster's waiting queue or delegated to a grid scheduler for its later allocation, either in the same or in a different cluster. If an RM receives a job from a GS, it has to accept it, even if that means adding it to the local waiting queue. Once the job is finished, the RM will send the result to the requester user. 
%describe the VGS application
\\\\
The system designed consists on 5 grid schedulers and 20 clusters containing 1,000 nodes each. The minimum workload the system is able to support is 10,000 jobs, for simplicity purposes each job demands only one processor for its execution. When a job arrives at a cluster it records the cluster's identifier, in case that the job is reallocated to a different cluster it will trace the additional identifiers of the clusters where it passes.
\\\\
The system is resilient to single failures. That is, the system is able to operate normally even if there is a failure in a GS or in an RM, the user is not be aware of these failures, her workload is carried out transparently. Lastly, the system handles with load imbalance. The ratio of the jobs arriving at the most and least loaded cluster the system is able to tolerate is 5:1.
