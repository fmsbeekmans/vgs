\section{Background on application}
The Virtual Grid Simulator (VGS) is a tool to monitor how a simulated distributed system performs conducting user workloads. The system consists on 5 grid schedulers (GS) and 20 clusters containing 1000 nodes each. On top of each cluster there is a resource manager (RM) which controls the cluster, thus, there are in total 20 RM. Additionally, several clusters are controlled by a grid scheduler. Users send their workload (jobs) to the cluster of their selection. If the selected cluster is free (there are nodes available to perfom the work), it will accept the job adding it to the local queue. In case that the queue is full, the RM will notify the GS. The latter will maintain another queue to allocate the jobs as soon as possible, either on the same or another. Once the job is finished, the RM will send the result to the requester user and release the resources. Once a RM receives a job from the GS, it has to accept it, even if that means adding it to the local waiting queue. The minimum workload the system support is 10,000 asynchronous stateless jobs of fixed duration. Each job records the identifier of the cluster when it originally arrives. In the case that job is reallocated to another cluster it will add the identifier of all the cluster where it passes.
\\\\
The system is resilient to single failures. That is, the system is able to tolerate failures in one grid scheduler node or one resource manager node. However, to increase the availability of the system and its tolerance to failures WantDS also covered the failure of two nodes, either two GS or one RM and one GS. The user is not aware of these failures, her workload is carried out transparently. To sum up, the system is distributed (multiple clusters), replicated (possibility of replicate the job in another node in case of failure), fault-tolerant (failures do not suspend the service) and scalable (possibility of add more clusters without changing the architecture). Also, the system load balance the job among the different clusters. The ratio of the jobs arriving at the most and least loaded cluster the system is able to support is 5. Lastly, to make debugging of the system easier, all events are logged in two grid schedulers.
